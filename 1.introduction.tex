\chapter*{\textbf{Introduction générale}}
\addcontentsline{toc}{chapter}{Introduction générale}

\par Ces dernières années, avec le développement continu de la science et de la technologie, la croissance du  nombre d'articles scientifiques a commencé à augmenter massivement, par conséquent, la recherche de références bibliographiques, lors de la rédaction d'un article scientifique, est un processus qui prend du temps. Plusieurs approches ont été proposées pour aider les chercheurs à acquérir des articles scientifiques pertinents et utiles grâce à l'énorme quantité d'informations (surcharge d'informations) disponibles sur Internet. De nombreux systèmes de recommandation (RS) sont utilisés pour aider les chercheurs, Bien que les méthodes classiques de RS ont obtenu des succès remarquables en fournissant des recommandations d'articles, cependant, elles souffrent encore de nombreux problèmes tels que le démarrage à froid et la rareté des données. Dans cette perspective, la recommandation de citation contextuelle fait l'objet de recherches depuis environ deux décennies. De nombreux chercheurs ont utilisé les données textuelles appelées phrase de contexte, qui entourent la balise de citation, et les métadonnées de l'article cible pour trouver la recherche appropriée. Cependant, le manque d'ensembles de données d'analyse comparative bien organisée et aucun modèle pouvant atteindre des performances élevées a rendu la recherche difficile. 

\par Dans ce modeste mémoire, nous proposons un modèle basé sur l'apprentissage en profondeur et le traitement du langage naturel ainsi qu'un ensemble de données bien organisé pour la recommandation de citations d'articles scientifiques tenant compte du contexte. Nous remarquons tout d'abord que nous pouvons utiliser les informations des deux coté de la citation, afin d'améliorer les performances des recommandations des articles sensibles au contexte. Sur la base de cette idée, pour traiter le contexte des deux cotés d'une citation dans un article cible. Pour cela, nous utilisons des représentations d'encodeur bidirectionnel à partir de transformateurs BERT\cite{ch2bert}, un modèle pré-entraîné de données textuelles. nous avons utilisé un ensemble de données (FullTextPeerRead) contenant des phrases de contexte pour les références citées et les documents papier et qui comprend environ 4898 articles et plus 17000 d'enregistrements de citations structurées. Fesant de lui un ensemble de donnée bien organisé pour une recommandation d'article tenant compte du contexte. Enfin, un site web a été construit pour la démonstration, qui présente notre travail de manière plus intuitive.

\par Notre mémoire s'articulera sur quatre chapitres, au cours de ces derniers nous allons d'abord découvrire lors du chapitre~\ref{chapitre1} les systèmes de recommandations, les types de classes d'alghorithmes principaux qui les composent. Dans le chapitre~\ref{chapitre2} quand à lui, nous allons à la fois découvrir le deep learning, voir le fonctionnement des réseaux de neurones et certaines architectures (DL) les plus connues. Mais aussi le traitement du language naturel et les systèmes de recommandation ainsi que leur association avec l'apprentissage en profondeur. au cours du chapitre~\ref{chapitre3} nous allons procéder à la conception de notre modèle qui se divisera en quatre phases. Enfin, le dernier chapitre~\ref{chapitre4} nous allons implémenté et évalué notre modèle pour avoir des recommendations dans une application web.
